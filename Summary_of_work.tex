\documentclass[12pt,a4paper]{article}

\setlength{\parindent}{0.1 in}
%\setlength{\parskip}{0.1 in}
\setlength{\oddsidemargin}{0.25 in}
\setlength{\evensidemargin}{-0.25 in}
\setlength{\topmargin}{-0.5 in}
\setlength{\textwidth}{7.0 in}
\setlength{\textheight}{9.5 in}
\setlength{\headsep}{0.45 in}

%\usepackage[fleqn]{amsmath}
%\usepackage{amsfonts,graphicx}
\usepackage{amsmath,amsfonts,graphicx}
\usepackage[fleqn]{mathtools}
\usepackage{setspace}
\usepackage[colorlinks=false, pdfborder={0 0 0}]{hyperref}
\usepackage[nottoc]{tocbibind}
\usepackage{tocloft}
\usepackage[outermargin=2 in]{geometry}
\usepackage{scrextend}
\usepackage{tensor}
\usepackage{cancel}
\usepackage{slashed}


%Adding `Appendix' to the appendices
\usepackage[toc,page]{appendix}

%Add a bullet point to description items
\usepackage{enumitem}

%Mathematics
\usepackage{braket}
\usepackage{ulem}
\usepackage{xcolor}
\usepackage[font={small,it}]{caption}

\bibliographystyle{unsrt}

% Packages that Gavin uses
\usepackage{url}
\usepackage[font=footnotesize]{caption}
\usepackage[font=footnotesize]{subcaption}
\usepackage[]{microtype}
\usepackage{balance}
\usepackage{cite}
\usepackage{lmodern}
\usepackage[T1]{fontenc}
\usepackage{doi}

% Nice typesetting of SI units
\usepackage{siunitx}
\sisetup{range-phrase=--}
\sisetup{separate-uncertainty = true}

% SImons packages
\usepackage{float}



%New commands
%Maths
\newcommand{\beq}{\begin{equation}}
\newcommand{\eeq}{\end{equation}}
\newcommand{\beqn}{\begin{equation*}}
\newcommand{\eeqn}{\end{equation*}}
\newcommand{\bea}{\begin{align}}
\newcommand{\eea}{\end{align}}
\newcommand{\p}{\partial}
\newcommand{\trace}[1]{\mathrm{Tr}\left[#1 \right]}
\newcommand{\ptrace}[2]{\mathrm{Tr}_{#1} \left[ #2 \right]}
\newcommand{\bpmat}{\begin{pmatrix}}
\newcommand{\epmat}{\end{pmatrix}}
\newcommand{\vv}[1]{\mathbf{#1}}
\newcommand{\mat}[1]{\uuline{#1}}
\newcommand{\norm}[1]{\| #1 \|}
\newcommand{\op}[1]{\mathbb{#1}}
\newcommand{\vhat}[1]{\hat{\vv{#1}}}

%Renewed commands, in order for them to take arguments with automatically adjusted brackets 
\renewcommand{\dim}[1]{\mathrm{dim}\left( #1\right)}
\renewcommand{\det}[1]{\mathrm{det} \left( #1 \right)}
\renewcommand{\exp}[1] {\mathrm{exp} \left[ #1 \right]}


%\mathbb Letters
\newcommand{\identity}{\mathbb{I}}
\newcommand{\inreal}{\mathbb{R}}
\newcommand{\incomplex}{\mathbb{C}}

%Redefine Braket
\renewcommand{\braket}[1]{\left\langle #1 \right\rangle}

%Integration
\newcommand{\intd} {\mathrm{d}}

%Operators
\newcommand{\phihat}{\hat{\phi}}
\newcommand{\xhat}{\hat{x}}
\newcommand{\phat}{\hat{p}}
\newcommand{\Dhat}{\hat{D}}
\newcommand{\Hhat}{\hat{H}}
\newcommand{\ahat}{\hat{a}}
\newcommand{\bhat}{\hat{b}}
\newcommand{\chat}{\hat{c}}
\newcommand{\Phihat}{\hat{\Phi}}

%Channels
\newcommand{\channel}[3]{\mathcal{#1}^{#2 \rightarrow #3}}


%Caligraphy letters
\newcommand{\cl}[1]{\mathcal{#1}}
\newcommand{\Hilbert}{\mathcal{H}}
\newcommand{\calN}{\mathcal{N}}
\newcommand{\Lag}{\mathcal{L}}
\newcommand{\calD}{\mathcal{D}}


%Pauli
\newcommand{\Xhat}{\hat{X}}
\newcommand{\Yhat}{\hat{Y}}
\newcommand{\Zhat}{\hat{Z}}
\newcommand{\PauliX}{\bpmat 0 & 1 \\ 1 & 0 \epmat}
\newcommand{\PauliZ} {\bpmat 1 & 0 \\ 0 & -1\epmat}



%Gell-Mann matrices
\newcommand{\GMone} {\bpmat 0 & 1 & 0 \\ 1 & 0 & 0 \\ 0 & 0 & 0 \epmat }
\newcommand{\GMsix}{\bpmat 0 & 0 & 0 \\ 0 & 0 & 1\\ 0 & 1 & 0\epmat}

%Density matrices
\newcommand{\rhotwo}{\bpmat 1 & e^{-it} \\ e^{it} & 1 \epmat}
\newcommand{\rhothree} {\bpmat 1 & e^{it} & e^{2it} \\
e^{-it} & 1 & e^{it} \\
e^{-2it} & e^{-it} & 1 \epmat}



%Misc
\def\dbar{{\mathchar'26\mkern-12mu d}} %a $d$ with a bar through its stem


\newcommand{\eq}[1]{$#1$}

%Undertilded quantities
\newcommand{\tildeq}{\underset{^\sim}q}
\newcommand{\tildep}{\underset{^\sim}p}

%Curly letters
\newcommand{\calE}{\mathcal{E}}

\newcommand{\Nhat}{\hat{N}}

%Wave vector shortening
\newcommand{\kvec}{\vv{k}}

% Commands that Gavin likes
% SUBSCRIPT COMMAND
\renewcommand{\d}[1]{\ensuremath{\operatorname{d}\!{#1}}}
\newcommand{\subscript}[1]{$_{\text{#1}}$}
\newcommand{\superscript}[1]{$^{\text{#1}}$}

% NEW DEFINITION OF SQRT - much neater, with tick at end
% ADAPTED FROM http://en.wikibooks.org/wiki/LaTeX/Mathematics
% rename \sqrt as \oldsqrt
\let\oldsqrt\sqrt
% define new \sqrt in terms of the old one
\def\sqrt{\mathpalette\DHLhksqrt}
\def\DHLhksqrt#1#2{%
	\setbox0=\hbox{$#1\oldsqrt{#2\,}$}\dimen0=\ht0
	\advance\dimen0-0.2\ht0
	\setbox2=\hbox{\vrule height\ht0 depth -\dimen0}
	{\box0\lower0.4pt\box2}}


% Nice typesetting of SI units
\usepackage{siunitx}
\sisetup{range-phrase=--}
\sisetup{separate-uncertainty = true}

% SImons packages
\usepackage{float}

% Make shiny tables
\usepackage{booktabs}









\begin{document}
\title{Summary of work}
\author{Gavin Dold, Sofia Qvarfort, Simon Schaal}
\maketitle
\tableofcontents
\section{Introduction}
We have simulated a probe qubit interacting with four data qubits. 


\section{Dephasing}
In this section, we present results from simulations of the Lindblad master equation where a dephasing channel is turned on. This channel contains the Lindblad operator
\beq
L  = \sqrt{\Gamma} \sigma_z
\eeq

where $\Gamma$ is the dephasing parameter, which is also $1/\tau$ where $\tau$ is the dephasing time. 

We evolve the system under the Lindblad master equation for an odd number of errors. This causes the final phase of the probe qubit to end up at $\phi = \pi$. 

In Figure \ref{fig:BlochsphereDephasing}, we see the effect of a small dephasing coefficient (100) whereas in Figure \ref{fig:somethig} we see the effect of larger dephasing (500). Note that the dephasing happens primarily as the interaction between the probe qubit and the data qubit is weak. 



\begin{figure}[!ht]
  \caption{A picture of a gull.}
  \centering
    \includegraphics[width=\textwidth]{Figures/Circ_orbit_odd_no_dephasing.png}
\end{figure}




\begin{figure}[!ht]
  \caption{A picture of a gull.}
  \centering
    \includegraphics[width=\textwidth]{Figures/Circ_orbit_odd_100_dephasing.png}
\end{figure}

In Figure \ref{fig:plot} we see the effect of dephasing on the probability of measuring the correct value of the probe qubit. Since there is an odd number of errors, we want the probe qubit to end up in the $\ket{-}$ state. However, the dephasing will cause the probe qubit to become a mixed state $\rho$, for which we have a finite probability of measuring $\ket{+} $ instead of $\ket{-}$. For complete dephasing, the probe qubit becomes a mixed state which has a 50--50\% chance of measuring either state. 

\begin{figure}[!ht]
	\caption{This is not a picture of a gull.}
	\centering
	\includegraphics[width=\textwidth]{Figures/measurement_dephasing_graph.png}
\end{figure}

In Figure \ref{fig:Plot}, we have market the data-point for dephasing corresponding to the decoherence time for Bismuth, which is one of the proposed donor types for the probe qubit. The decoherence time of bismuth is \SI{2.7}{\second} \cite{something}, which leads to a dephasing parameter of \SI{0.37}{\per\second}. 



\clearpage
\section{Data qubit displacement}
The effect of displacement of the data qubits from the ideal was investigated. Ideally the data qubits would be in a square lattice of spins precisely $D = \SI{400}{\nano\metre}$ apart, but due to inaccuracies in dopant spin placement each qubit will have small displacements from the ideal lattice position.

This is modelled by generating a uniform random displacement within a given pillbox $xy$-radius and $z$-height. Simulations from the original paper show $\textrm{radius} = \textrm{height} = \SI{6}{\nano\metre}$ to be a threshold for this scheme. The phase accumulated over 25 runs for this pillbox size is plotted as a histogram in fig.\@ \ref{fig:DisplacementPhaseHistogram}, showing a maximum phase error of $\tfrac{\pi}{4}$ for these runs. 



The effect of displacements in the $x$-$y$ plane and the $z$-axis are significantly different in magnitudes, due to the $\tfrac{1}{d^3}$ term in the Hamiltonian being most strongly affected by $z$ displacements. This effect was investigated by artificially setting displacements in these directions.

Fig.\@ \ref{fig:zoffset} shows changes in accumulated phase due to $z$-displacements. The first 2 qubits are displaced \SI{4}{\nano\metre} down, slowing the evolution and giving a noticeable phase error after half a cycle. However, qubit 4 is displaced \SI{3}{\nano\metre} upwards, reducing $d$ and resulting in faster evolution. The effect is a small phase error from the ideal $2\pi$.

Fig.\@ \ref{fig:inwarddisplacement} shows the effect of displacements in the $x$-$y$ plane. For this run, all data qubits were displaced \SI{10}{\nano\metre} inwards with respect to the circular motion. The phase error on each individual qubit is then less than that produced by the \SI{3}{\nano\metre} $z$-displacement of fig.\@ \ref{fig:zoffset}, showing the smaller sensitivity to displacement in the $x$-$y$ plane, though the overall error after all 4 qubits is greater as in the $z$-direction, +$z$ and --$z$ errors cancel out somewhat, whereas $xy$ displacement errors will always slow the evolution.

\begin{figure}
	\centering
	\includegraphics[width=\textwidth]{figures/Displacement_phase_histogram.pdf}	
	\caption{Phase errors over 25 runs as a result of randomly generated data qubit displacements within a pillbox of half-height \SI{3}{\nano\metre} and radius \SI{6}{\nano\metre}. These values are a threshold for the proposed scheme.}
	\label{fig:DisplacementPhaseHistogram}
\end{figure}

\begin{figure}
	\centering
	\caption{Phase errors as a result of misplaced data qubits.}
	\begin{subfigure}[t]{\textwidth}
		\centering
		\includegraphics[width=0.7\textwidth]{figures/z_offset.pdf}
		\caption{Evolution of probe qubit with data qubit displacement in the Z direction. 1st and 2nd qubits are displaced \SI{4}{\nano\metre} down, slowing the evolution, the 4th is displaced \SI{3}{\nano\metre} up with a resultant increase in phase accumulated. The overall deviation from $2\pi$ is small as a result.}
		\label{fig:zoffset}
	\end{subfigure}
	\begin{subfigure}[t]{\textwidth}
		\centering
		\includegraphics[width=0.7\textwidth]{figures/10nm_displacement_inward.pdf}
		\caption{Phase accumulated when all 4 data qubits are displaced \SI{10}{\nano\metre} towards the centre of the circle. Other directions of \SI{10}{\nano\metre} displacements result in similar phase deviations.}
		\label{fig:inwarddisplacement}
	\end{subfigure}
	\label{fig:overall_displacement}
\end{figure}


\section{Path Jitter}

\end{document}