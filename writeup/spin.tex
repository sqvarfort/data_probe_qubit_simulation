
\subsection{Spin species}

Having demonstrated how the parity measurement is realised and how long each measurement could take we will now look into different available spin systems. 

As shown in the previous sections we require \hbox{$\Delta d^3/ J > 10^4$}. This can be achieved by choosing two different spin species for the data and probe qubits. Any sensible combination of species given in table.\@ \ref*{TAB:qubits} fulfils this criterion.

For the data qubits we desire long coherence times to maintain the quantum information throughout a large number of parity measurements. Furthermore, we only require global control for the data qubits during operation as logical operations on the encoded qubits can be performed using the stabiliser measurements.
For of the probe qubits we merely require coherence times which are longer than the time it takes to perform a parity measurement. In addition, readout is important and individual control is required.

\bgroup
\def\arraystretch{1.3}%
\begin{table}[H] 
	\footnotesize 
	\begin{tabular}{lrrrr}
		\hline
		Spin qubit & $T_1$ & $T_2^{*}$ & $T_2$ & $T_{2,\textrm{decoupl}}$ \\ \hline 
		P (nat. Si, mK, SET) \cite{Pla2012}& $0.7\, $s & $^{10}55\, $ns  & $^5206\, \si{\micro s}$ & $410\, \si{\micro s}$  \\
		P ($^{28}$Si, mK, SET) \cite{Muhonen2014}&  & $^7160\, \si{\micro s}$  & $^41\, $ms & $560\, $ms \\
		P$^{\text{nuc}}$ ($^{28}$Si, mK, SET) \cite{Muhonen2014}& & $500\, \si{\micro s}$ & $1.75\, $s & $35.6\, $s \\
		P ($^{28}$Si, $6.9\, $K, bulk) \cite{Morley2010}& &  & $14\, $ms &  \\
		P ($^{28}$Si, $1.8\, $K, bulk) \cite{Tyryshkin2011}& &  & $0.6\, $s &  \\
		Bi ($^{28}$Si, $4.3\, $K, bulk CT) \cite{Wolfowicz2013} & $9\, $s &  & $^12.7\, $s &\\
		NV ($^{12}$C, RT) \cite{Balasubramanian2009,Bar-Gill2013} & & & $^21.8\, $ms & $3.3\, $ms \\
		NV ($^{12}$C, $77\, $K) \cite{Bar-Gill2013} & & &  & $0.6\, $s \\
		SiC ($20\, $K) \cite{Christle2014} & & $^81.1\, \si{\micro s}$ & $^31.2\, $ms &  \\
		SiC (RT) \cite{Koehl2011} & $185\, \si{\micro s}$ & $^9214\, $ns & $^640\, \si{\micro s}$ &   \\
		\hline
	\end{tabular} 
	\caption{Summary of coherence times of various spin qubit systems. The spin species as well as the measurement temperature is given. RT stands for room temperature while CT refers to clock transitions. For donors in Silicon we distinguish between near surface dopants readout using a single-electron-transistor (SET) and bulk dopants. The numbers in superscript refer to fig.\@ \ref{fig:dephasingplot}.}
	\label{TAB:qubits}
\end{table}
\egroup




This illustrates that coherence is an important parameter for choosing a spin system. Dephasing is a reversible loss of coherence due to inhomogeneous broadening characterised by the dephasing time $T_2^*$ which is usually much shorter than a millisecond (see table\@ \ref{TAB:qubits}). Luckily we should be able to combine the parity measurement with spin echo or even dynamical decoupling techniques enhancing coherence to the more generous $T_2$ and $T_{2,\textrm{decoupl}}$ times which represent irreversible loss of coherence.

Donors in silicon are excellent candidates for both data and probe qubits as they offer extremely long coherence times when implanted in purified $^{28}$Si (see table\@ \ref{TAB:qubits}). Moreover, high-fidelity electrical readout using a single-eletron-transistor (SET) \cite{Pla2012,Pla2013,Muhonen2014} as well as optical \cite{Lo2015} readout has been demonstrated for $^{31}$P donors and is feasible for $^{209}$Bi. Control of individual spins can be achieved by tuning them in and out of resonance using the Stark effect \cite{Pica2014}. The control electronics have a small footprint making a large scale implementation feasible. Being able to transfer the electron spin state to the nuclear spin offers even longer coherence times. Additionally, $^{209}$Bi has several clock transitions (CT) which are insensitive to magnetic fields, leading to a very long electron spin coherence time $T_2=\SI{2.7}{\second}$. However, operation in this regime will make the $^{209}$Bi qubit also insensitive to the magnetic dipole-dipole interaction. Tuning in an out of the CT will be necessary to implement the parity measurement. Proof-of-concept implementations could benefit from the fact that donors in silicon offer moderate coherence times even at elevated temperatures. However, the coherence time demonstrated in bulk donors still remain to be reproduced for near-surface donors as required for this scheme.
Finally, MEMS devices with a sensitivity of \SI{25}{\hertz\per nm} and $0.3\, $nm accuracy \cite{Chu2003} as well as an accuracy of $5\, $nm with a bandwidth of $30\, $Hz have been demonstrated. This gives good prospects for realisation of highly accurate and fast MEMS devices as needed for this scheme in the near future. 


Besides donors in Si we imagine nitrogen vacancies (NV) to be a promising candidate for proof-of-concept demonstrations. NV centres are highly developed qubits \cite{Bar-Gill2013} which offer good coherence times even at room temperature (RT) \cite{Balasubramanian2009}. Individual addressing can be achieved using the Stark effect. Optical readout and the possibility of integrating NV centres into an atomic force microscope (AFM) tip makes NV centres good candidates for a probe qubit \cite{Grinolds2013}. Unfortunately, the NV centre readout frequency lies above the band-gap of Si resulting in carrier excitation. Thus, readout has to be performed carefully in safe distance to the data qubits. Additionally diamond MEMS devices have yet to be demonstrated.

Recently, defects in SiC have attracted lots of attention \cite{Morello2015} as coherent manipulation of individual spins with predicted room temeprature coherence times on the order of milliseconds has been demonstrated \cite{Widmann2014}. At cryogenic temperatures ($20\, $K) such long spin coherence has already been achieved \cite{Christle2014}. Current experiments still suffer from a low collection efficiency. Having near infra-red transitions, SiC shows excellent prospects for integration with optical fibres. Moreover, AFM tip integration is feasible. However, SiC is still at a very early research stage.  


