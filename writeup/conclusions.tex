
\section{Conclusions and outlook } \label{sec:conclusions}
In this report, we we have presented results obtained from simulating a system of one probe qubit and four data qubits based on the setup proposed in \cite{OGorman2016}. We investigated how physically motivated errors affected the performance of the system and were able to determine the desired values for certain experimental parameters. 

We found that the ratio between the coupling constant $J$ and the detuning $\Delta$ should be of the order $\Delta r^3/J \ge 10^4$ to avoid flip-flopping of the data qubits. Furthermore, we characterised the impact of changing physical parameters, such as the separation between the probe and data qubits $d$ and the in-plane qubit lattice constant $D$ on the parity measurement time.  
Looking at the introduction of errors, our results show that path jitter has little effect on the overall performance, meaning that the resulting path jitter stays within the $4.4\%$ threshold set out in \cite{OGorman2016}. On the other hand, we find that dephasing has a strong influence on the success probability $p_{succ}$ of measuring the probe qubit in the correct state. Assuming that $p_{succ}$ is the largest of all probabilities for failure, out of all spin species we investigated, only bismuth was found to have dephasing times short enough to push $p_{succ}$ above the surface code threshold. Furthermore, introducing a random displacement in the position of the data qubits yielded $p_{succ} = 98.? \%$, which  again is above the surface code threshold. These results are obtained even when using values for the displacement error that are below the thresholds, as shown in \cite{OGorman2016}. One is however right to question whether the comparison between our results and those obtained by O'Gorman \textit{et al}. hold. Whereas they are looking at the fault-tolerance of an entire system, we are investigating the stabiliser measurement in isolation. One could simply compensate for the failing stabiliser by performing the measurement twice or trice, although this would severely increase the time required for each error-correcting step. More investigation is needed to see whether the effect of the data qubit displacement is as severe as shown here. 






As to how the project can be continued, there are many avenues that are worthy of investigation. We wrote the code for simulating qubit initialisation errors, but were unable to investigate the resulting effect due to constraints on computational time. Another rather natural extension is to combine the errors presented above and see how they affect the total performance. Even so, the dephasing time is likely to be the single most important factor that influences the probability of successfully distinguishing the probe qubit state. Therefore, it is imperative that we explore ways to decrease the impact of dephasing. We can identify many ways to do so, some already mentioned in Section \ref{sec:dephasing}. The first would be to investigate whether the full orbit measurement can be performed in a shorter time, especially in the case of the circular orbit. This can be done by decreasing the separation $d$ between the probe and data qubits or by decreasing the lattice parameter $D$ of the data and probe qubits. A shorter parity measurement time could also be achieved by speeding the probe qubit orbit up when being far away from the data qubits. 
There are however two limit factors in these approaches. The first is that the speed and precision achievable by the MEMS will be limited. Current MEMS will not be able to achieve near $1\, $nm precision at a speed of $1\, $kHz \cite{Koo2012,Chu2003} but we are optimistic about improvements in the near future. The second is whether the increased proximity to other data qubits induces interference and cross-talk between the probe qubit and other, neighbouring data qubits. 
An alternative to decreasing the time for one cycle is to tune the probe qubits in and out of a protected states such as clock transitions for bismuth or nuclear spin states (see Table.\@ \ref{TAB:qubits}). In Figure \ref{fig:dephasing} we saw that the clock transition in bismuth has a long dephasing time as it is protected from magnetic field noise. However, this makes the transition insensitive to any qubit interaction making it necessary to tune the probe qubit in and out of the clock transition to perform the parity measurement. The feasibility of this tuning on the order of tenths to hundreds of microseconds remains to be investigated.

Furthermore, we have assumed that the 4-qubit system is completely shielded from the influence of neighbouring data qubits from other quadrants. It would be interesting to simulate larger systems, for example a $4\times 4$ grid of data qubits and four probe qubits to ensure that no cross-talk occurs between the data qubits. One could then also investigate the effects of changing the lattice constant $D$. 

Finally, one can investigate whether it ultimately should be the data qubits or the probe qubits that  move. In our simulation, we only take into account the changing relative distance, and so no asymmetry arises due to swapping the data qubits with the probe qubits. There are however advantages in moving the data qubit stage instead of the probe qubit stage. Since the probe qubits need local addressing for measurements and initialisation, they will require wiring and control infrastructure to be positioned on the slab. One can image that moving the data qubits instead will be an easier experimental task. 

With our results, we hope to have provided a deeper understanding of how a surface code silicon quantum computer might be realised. The results from our simulations seem promising and we hope that they can prove an aid for future attempts to realise this system. 













