
\section{Conclusions and outlook } \label{sec:conclusions}
In this report, we we have presented results obtained from simulating a system of one probe qubit and four data qubits based on the setup proposed in \cite{the paper}. We found that physically motivated errors of real-world magnitude did not severely impact the performance of the device, and both the phase error and reduced stabiliser measurement probabilities were within the thresholds set out in \cite{the paper!!}. We found that the ratio between the coupling constant $J$ and the detuning $\Delta$ should be of the order $\Delta r^3/J = 10^4$ to avoid the data qubits displaying flip-flopping behaviour. Furthermore, we characterised the impact on the of changing physical parameters, such as the distance between the probe and data qubits. Out of the possible candidate qubit species, bismuth displayed the best coherence times and opens up some interesting options due to its clock transition. It remains to be seen whether one can Dephasing turned out to be the most crucial parameter for the performance of the system To summarise, our simulations show that realistic errors that arise in an experimental implementation stay below the thresholds for fault-tolerance. 

Many avenues remain to be investigated. We wrote the code for simulating qubit initialisation errors, but were unable to investigate the resulting effect due to constraints on computational time. Another rather natural extension is to combine the errors presented above and see how they affect the total performance. Even so, the dephasing time is likely to be the single most important factor that influences the probability of successfully distinguishing the probe qubit state. Therefore, it is imperative that we explore ways to decrease the impact of dephasing. We can identify many ways to do so, some already mentioned in Section \ref{sec:dephasing}. The first would be to investigate whether the full orbit measurement can be performed in a shorter time, especially in the case of the circular orbit. This can be done by decreasing the distance $d$ between the probe and data qubits. Due to the $\frac{1}{r^3}$ term in the Hamiltonian, even a small displacement will significantly alter the interaction strength. If $d$ decreases, the interaction strength increases and the time required for a full orbit will go down. There are however two limit factors in this approach. The first is the speed achievable by the MEMs that will move the probe qubit. Current speeds are about --- \cite{something} which is enough [CHECK THIS] for current speeds. The second is whether the increased proximity to other data qubits induces interference and cross-talk between the probe qubit and other, neighbouring data qubits. 
 An alternative to decreasing the time for one cycle is to tune the probe qubits in and out of a protected clock transition \cite{something}. In Figure \ref{fig:dephasing} we saw that the clock transition in bismuth has a long dephasing time and is protected from magnetic field noise. It makes the transition unsuitable for the actual interaction, but if one could tune the probe qubit in and out of the clock transition the dephasing time could be significantly increased. 

Furthermore, we have in our simulation assumed that the 4-qubit system is completely shielded from the influence of neighbouring data qubits from other quadrants. It would be interesting to simulate larger systems, for example a $4\times 4$ grid of data qubits and four probe qubits to ensure that no cross-talk occurs between the data qubits. One could then also investigate the effects of changing the lattice constant $D$. 

Finally, one can investigate whether it ultimately should be the data qubits or the probe qubits that  move. In our simulation, we only take into account the changing relative distance, and so no asymmetry arises due to swapping the data qubits with the probe qubits. There are however advantages in moving the data qubit stage instead of the probe qubit stage. Since the probe qubits need local addressing for measurements and initialisation, they will require wiring and control infrastructure to be positioned on the slab. One can image that moving the data qubits instead will be an easier experimental task. 

With our results, we hope to have provided a deeper understanding of how a surface code silicon quantum computer might be realised. The results from our simulations seem promising and we hope that they can prove an aid for future attempts to realise this system. 













