
\section{Conclusions and outlook } \label{sec:conclusions}
In this report, we have presented the results obtained from simulating the interaction between one probe qubit and four data qubits. This system is a fundamental building block in the proposed scheme in \cite{the paper}. We found that physically motivated errors of real-world magnitude did not severely impact the performance of the device, and both the phase error and reduced stabiliser measurement probabilities were within the thresholds set out in \cite{the paper!!}. We were able to gain an estimate for the magnitude of required physical parameters, such that the ratio between the coupling constant $J$ and the detuning $\Delta$ be $\Delta r^3/J = 10^4$. We also characterised the impact of physical parameters such as the distance between the probe and data qubits on the correct evolution time, as well as reviewed the various material choice for probe and data qubits. Out of the possible candidate qubit species, we found that bismuth provided coherence times good enough for implementation. Given all the errors, we found that dephasing is by far the most crucial parameter that will impact the performance of our system. 




Many avenues remain to be investigated. We wrote the code for simulating qubit initialisation errors, but we were not able to investigate their effect due to constraints on computational time. The other clear aspect to explore is to combine the errors presented above and see how they affect the total performance. Even so, the dephasing time is likely to be the single most important factor that influences the probability of successfully identifying the probe qubit state. Therefore, it is imperative that we explore ways to decrease the impact of dephasing. We can identify many ways to do so, some already mentioned in Section \ref{sec:dephasing}. The first would be to investigate whether the measurement can be performed in less time, especially with regard to the circular orbit. This can be done by decreasing the distance $d$ between the probe and data qubits. Due to the $\frac{1}{r^3}$ term in the Hamiltonian, even a small displacement will significantly alter the interaction strength. A smaller $d$ will increase the interaction strength, and the time needed for orbit will decrease. There are two limiting factors for this implementation. The first is the speed that is achievable by using MEMs to move the probe qubit. The second is whether the increased proximity to other data qubits might cause cross-talk between them. This is however unlikely again due to the $\frac{1}{r^3}$ term in the Hamiltonian, but it should nevertheless be carefully considered. An alternative to decreasing the time for one cycle is to tune the probe qubits in and out of a protected clock transition \cite{something}. In Figure \ref{fig:dephasing} we saw that the clock transition in bismuth has a long dephasing time and is protected from magnetic field noise. It makes the transition unsuitable for the actual interaction, but if one could tune the probe qubit in and out of the clock transition the dephasing time could be significantly increased. 

Another natural venue is to simulate larger systems, for example a $4\times 4$ grid of data qubits with four probe qubits. One could then explore the effects of neighbouring spins and to what degree the lattice constant $D$ can be altered. 

Finally, one last aspect is whether it is the data qubits or the probe qubits that should be moved. In our simulation, we only take into account the changing relative distance, and so no asymmetry arises due to swapping the data qubits with the probe qubits. Since the probe qubits need local addressing for measurements and initialisation, which in turn will require plenty of wires and control structures, one can imagine that it is easier to experimentally implement the movement of the data qubits. 

With our results, we hope to have provided a deeper understanding of how a surface code silicon quantum computer might be realised. The results look promising and we hope that our simulation can provide the necessary details to within the near future realise such a quantum system. 